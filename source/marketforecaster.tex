\chapter{MarketForecaster (Console Application)}
\addcontentsline{toc}{chapter}{MarketForecaster (Console Application)} % TODO : Necessary?

\texttt{MarketForecaster} is a console application for forecasting markets using simple time series prediction. It is designed to be as lightweight as possible, delegating all the "grunt work" to the underlying NeuralNetwork class library.

%%%%% REQUIREMENTS DECUMENTATION %%%%%
\section{Requirements Specification}

\textbf{MarketForecater} is a market forecasting console application. It enables the user to
\begin{itemize}
\item build raw time series by loading them from a file,
\item preprocess the loaded time series (e.g. scale or log it),
\item forecast (raw or preprocessed) time series using different teachers, and
\item execute forecasting sessions, and
\item log the results of the forecasting session into a file.
\end{itemize}

\subsection{Supported Features}

Currently, only \textit{one-step-ahead forecasts} are supported, i.e. at any given moment in time, only the immediate next time series element can be predicted. However, should the \textit{multistep-ahead forecasts} be required, there exists an intuitive way to simulate them using one-step-ahead forecasts. We just need to run one-step-ahead forecast for each step we want to obtain, and feed the output back into the network between each two consecutive runs, i.e. replace the lag 1 input of the later run with the output of the earlier run. Naturally, this workaround is not by any means a full substitute for the multistep-ahead forecasting, as we are are using data already flawed with predictive inaccuracies to predict further data, thus cumulating the forecasting error. 

%%%%% DESIGN AND ARCHITECTURE DOCUMENTATION %%%%%
\section{Design and Architectre Documentation}

%%%%% TECHNICAL DOCUMENTATION %%%%%
\section{Technical Documentation}

\textbf{MarketForecaster} is a class library written in C\# programming language (version 3) and targeted for Microsoft .NET Framework (version 3.5). It is being developed in Microsoft Visual C\# 2008 Express Edition as a part of the \textbf{MarketForecaster} colution - an entire solution aimed at forecasting markets using artificial neural networks.

\subsection{Dependencies}

\textbf{MarketForecaster} uses the "in-house" developed \textbf{NeuralNetwork} class library.

\subsection{Using \textbf{MarketForecaster}}

To use MarketForecaster, ...

%%%%% END USER DOCUMENTATION %%%%%
\section{End User Documentation}

\subsection{Step 1 : Building the Time Series}

First and foremost, it is necessary to specify the file from which the time series should be loaded - the \texttt{.timeseries} file. The file is allowed to contain only real numbers separated by a \textit{single space character}. Whether they are all stored in a single line or each on a separate line, or whether the file contains blank lines is irrelevant. This leniency regarding the \texttt{.timeseries} file format allows for an arbitrary tabulation of contained data, e.g. to be easily interpreted by a human reader.
Taking our \texttt{airline\_data.timeseries} file as an example, it contains 12 rows (representing years from 1941 to 1960) each consisting of 12 real values (each representing the monthly total of international airline passengers).

\noindent \texttt{RawTimeSeries rawTimeSeries = new RawTimeSeries( "airline\_data.timeseries" );}

The time series is now loaded and ready to be (preprocessed and) forecast.

\subsection{Step 2 : Preprocessing the Time Series}

Optionally, we may want to preprocess the time series before we attempt to forecast it. In many situations, this step becomes a compulsory one, as it is impossible to reasonably forecast the raw time series.

\noindent \texttt{double scalingFactor = 0.01;//
ScaledTimeSeries scaledTimeSeries = new ScaledTimeSeries( rawTimeSeries, scalingFactor );}

Instead of simply scaling the data, the user may elect to transform the data by taking the (natural) logarithms of the values.

In any case, the data is now preprocessed (scaled or logged) and ready to be forecast.

\subsection{Step 3 : Specifying the Teacher}

The default teacher to be used during the training session is the Backpropagation teacher. However, it is easily replaceable by any other teacher the underlying NeuralNetwork class library supports.

\subsection{Step 4 : Specifying the Sets of Lags and the Numbers of Hidden Neurons for the Forecasting Session}

\subsection{Step 5 : Executing the Forecasting Session}